\documentclass[11pt, aspectratio=169]{beamer}
\usetheme{default}

\usepackage[utf8]{inputenc}
\usepackage[german]{babel}
\usepackage{graphicx}
\usepackage{tikz}
\usepackage[absolute,overlay]{textpos}
\usepackage{gensymb}

\setlength{\TPHorizModule}{1cm}
\setlength{\TPVertModule}{1cm}

%\author{JeanHilftDir}
\title{Mathe Nachhilfe}
\subtitle{Analytische Geometrie - Vektoren}
%\logo{}
%\institute{}
\date{}
%\subject{}
%\setbeamercovered{transparent}
\setbeamertemplate{navigation symbols}{}

\definecolor{lightblue}{RGB}{55,96,146}
\newcommand{\coord}{
		% draw help lines
		%\foreach \x in {.5,1,1.5,2,2.5,3,3.5,4,4.5,5}
		%	\draw[style=help lines] (\x,0,0) -- (\x,0,5);
		%\foreach \x in {.5,1,1.5,2,2.5,3,3.5,4,4.5,5}
		%	\draw[style=help lines] (\x,0,0) -- (\x,5,0);
		%\foreach \y in {.5,1,1.5,2,2.5,3,3.5,4,4.5,5}
		%	\draw[style=help lines] (0,\y,0) -- (0,\y,5);
		%\foreach \y in {.5,1,1.5,2,2.5,3,3.5,4,4.5,5}
		%	\draw[style=help lines] (0,\y,0) -- (5,\y,0);
		%\foreach \z in {.5,1,1.5,2,2.5,3,3.5,4,4.5,5}
		%	\draw[style=help lines] (0,0,\z) -- (0,5,\z);
		%\foreach \z in {.5,1,1.5,2,2.5,3,3.5,4,4.5,5}
		%	\draw[style=help lines] (0,0,\z) -- (5,0,\z);
		
		% draw Axes
		\draw[thick,->] (0,0,0) -- (5,0,0) node[below]{$y$};
		\draw[thick,->] (0,0,0) -- (0,5,0) node[left]{$z$};
		\draw[thick,->] (0,0,0) -- (0,0,5) node[left]{$x$};
		
		% draw labels
		\draw (0,0,0) -- (0,-.1,0) node[below]{$\scriptstyle0$};
		\foreach \x in {1,2,3,4}
			\draw (\x,-.1,0) -- (\x,.1,0) node[below=4pt] {$\scriptstyle\x$};
		\foreach \y in {1,2,3,4}
			\draw (-.1,\y,0) -- (.1,\y,0) node[left=4pt] {$\scriptstyle\y$};
		\foreach \z in {1,2,3,4}
			\draw (-.1,.1,\z) -- (.1,-.1,\z) node[right,below] {$\scriptstyle\z$};
	}
\newcommand{\lastslide}{
		\begin{frame}
			\only<2->{
				\begin{textblock}{5}(0.5,3.5)
					\includegraphics[width=5cm]{graphix/subpointer.png}
				\end{textblock}
			}
			\only<3->{
				\begin{textblock}{7}(8.5,0)
					\begin{flushright}
						\color{lightblue}\Large
						facebook.com/JeanHilftDir\\
						Skype: JeanHilftDir
					\end{flushright}
				\end{textblock}
				\begin{textblock}{9}(6.5,4)
					\begin{flushright}
						\color{lightblue}\Large
						Folien: github.com/JeanHilftDir/Mathe\\
						
					\end{flushright}
				\end{textblock}
			}
		\end{frame}
	}
\AtBeginSection[]{
		\begin{frame}{Analytische Geometrie - Vektoroperationen}
			\tableofcontents[currentsection]
		\end{frame}
	}

\begin{document}
	\maketitle
	\begin{frame}{Analytische Geometrie - Vektoroperationen}
		\tableofcontents
	\end{frame}
	\section{Null-, Einheits-, Gegenvektor, Kartesisches Koordinatensystem}
	\begin{frame}{Nullvektor, Einheitsvektor, Gegenvektor, Kartesisches Koordinatensystem}
		\textbf{Nullvektor:}
		\uncover<2-> {$\vec{o} = \left( \begin{array}{c} 0 \\ 0 \\ 0 \end{array} \right)$}\\
		
		\ \\\textbf{Einheitsvektor:}
		\uncover<3-> {Vektor mit $\left| \vec{v} \right| = 1 $}\\
		\uncover<4-> {$\vec{x} = \left( \begin{array}{c} 1 \\ 0 \\ 0 \end{array} \right)$}
		\uncover<4-> {$\vec{y} = \left( \begin{array}{c} 0 \\ 1 \\ 0 \end{array} \right)$}
		\uncover<4-> {$\vec{z} = \left( \begin{array}{c} 0 \\ 0 \\ 1 \end{array} \right)$}\\
		\ \\\textbf{Gegenvektor zu $\vec{a}$:}
		\uncover<5->{$-\vec{a}$}\\
		\ \\\textbf{Kartesisches Koordinatensystem\uncover<1-6>{:}}\uncover<6->{(auch Orthonormiertes)\textbf{:}}\\
		\uncover<7->{Koordinatensystem $(O,\vec{x},\vec{y},\vec{z})$ mit}\\
		
		\begin{itemize}
			\item<8->$\left| \vec{x} \right| = \left| \vec{y} \right| = \left| \vec{z} \right| = 1 $
			\item<9->$\sphericalangle(\vec{x},\vec{y}) = \sphericalangle(\vec{y},\vec{z}) = \sphericalangle(\vec{z},\vec{x}) = 90^\circ $
			\item<10->$\vec{x},\vec{y} und \vec{z}$ bilden in dieser Reihenfolge ein Rechtssystem (Rechte-Hand-Regel).
		\end{itemize}
	\end{frame}
	\begin{frame}{Aufgaben}
		\begin{itemize}
			\item Geben Sie den Nullvektor an.
			\uncover<2->{$\vec{0} = \left( \begin{array}{c} 0 \\ 0 \\ 0 \end{array} \right)$}
			\item Sind folgende Vektoren Einheitsvektoren?\\
			$\vec{a} = \left( \begin{array}{c} 0 \\ 0 \\ 1 \end{array} \right)$
			\color{green}
			\uncover<3->\checkmark
			\color{black}\ 
			$\vec{b} = \left( \begin{array}{c} 1 \\ 1 \\ 1 \end{array} \right)$\ \ \ \ 
			$\vec{c} = \left( \begin{array}{c} \sqrt{2} \\ 0 \\ \sqrt{2} \end{array} \right)$\ \ \ \ 
			$\vec{d} = \left( \begin{array}{c} \sqrt{\dfrac{1}{3}} \\ \sqrt{\dfrac{1}{3}} \\ \sqrt{\dfrac{1}{3}} \end{array} \right)$
			\color{green}
			\uncover<3->\checkmark
			\color{black}
			\item Geben Sie die Gegenvektoren zu $\vec{a}, \vec{b}$ und $\vec{c}$ an.\\
			\uncover<4->{$-\vec{a} = \left( \begin{array}{c} 0 \\ 0 \\ -1 \end{array} \right)$}
			\uncover<4->{$-\vec{b} = \left( \begin{array}{c} -1 \\ -1 \\ -1 \end{array} \right)$}
			\uncover<4->{$-\vec{c} = \left( \begin{array}{c} -\sqrt{2} \\ 0 \\ -\sqrt{2} \end{array} \right)$}
		\end{itemize}
	\end{frame}
	\lastslide
	\maketitle
	\section{Grundrechenarten}
	\subsection{Addition, Subtraktion}
	\begin{frame}{Addition, Subtraktion}
		\begin{columns}[T]
			\column{.55\textwidth}
			$\vec{a} = \left( \begin{array}{c} 4 \\ 3 \\ 0 \end{array} \right)$
			$\vec{b} = \left( \begin{array}{c} 2 \\ -2 \\ 2 \end{array} \right)$
			\begin{flushleft}
				\begin{tikzpicture}[z=-.5cm]
					\draw[color=red,->] (0,0,0) -- (3,0,4) node[above=5pt,right]{$\vec{a}$};
					\draw[color=green,->] (0,0,0) -- (-2,2,2) node[above=5pt,right]{$\vec{b}$};
					\draw<2->[color=green,dashed, ->] (3,0,4) -- (1,2,6) node[below=10pt,right]{$\vec{b}$};
					\draw<7->[color=red, dashed, ->] (-2,2,2) -- (1,2,6) node[above=5pt,left=5pt]{$\vec{a}$};
					\draw<8->[color=green,dashed,->] (3,0,4) -- (5,-2,2) node[below=5pt,left=5pt]{-$\vec{b}$};
					\draw<3->[color=blue,->] (0,0,0) -- (1,2,6) node[above=5pt,right=15pt]{$\vec{a} + \vec{b}$};
					\draw<9->[color=cyan,->] (0,0,0) -- (5,-2,2) node[above=30pt,left]{$\vec{a} - \vec{b}$};
				
				\end{tikzpicture}
			\end{flushleft}
			\column{.45\textwidth}
			\uncover<4->{$\vec{a} + \vec{b} = \left( \begin{array}{c} x_a + x_b \\ y_a + y_b \\ z_a + z_b \end{array} \right)$}
			\uncover<5->{$ = \left( \begin{array}{c} 4 + 2 \\ 3 - 2 \\ 0 + 2 \end{array} \right)$}\\
			\uncover<6->{$\vec{a} + \vec{b} = \left( \begin{array}{c} 6 \\ 1 \\ 2 \end{array} \right)$}\\
			\ \\\uncover<10->{$\vec{a} - \vec{b} = \vec{a} + \left( -\vec{b}\right) $}
			\uncover<11->{$ = \left( \begin{array}{c} x_a + \left( -x_b \right) \\ y_a + \left( -y_b \right) \\ z_a + \left( -z_b \right) \end{array} \right)$}\\
			\uncover<12->{$\vec{a} - \vec{b} = \left( \begin{array}{c} x_a - x_b \\ y_a - y_b \\ z_a - z_b \end{array} \right)$}
			\uncover<13->{$ = \left( \begin{array}{c} 4 - 2 \\ 3 + 2 \\ 0 - 2 \end{array} \right)$}
			\uncover<14->{$\vec{a} - \vec{b} = \left( \begin{array}{c} 2 \\ 5 \\ -2 \end{array} \right)$}\\
		\end{columns}
	\end{frame}
	\subsection{Multiplikation mit Skalar / Skalierung}
	\begin{frame}{Multiplikation mit Skalar / Skalierung}
		\begin{columns}[T]
			\column{.55\textwidth}
			$\vec{c} = \left( \begin{array}{c} 4 \\ 3 \\ 1 \end{array} \right)$
			\begin{flushleft}
				\begin{tikzpicture}[z=-.5cm]
				\draw[color=red,->] (0,0,0) -- (3,1,4) node[above=5pt,right]{$\vec{c}$};
				\draw<2->[color=green,->] (0,0,1) -- (6,2,9) node[above=5pt,right]{$2\vec{c}$};
				\draw<4->[color=blue,<-] (0,0,2) -- (9,3,14) node[above=5pt,right]{$-3\vec{c}$};
				\draw<3->[color=cyan,->] (0,0,3) -- (1.5,0.5,5) node[above=5pt,right]{$\frac{1}{2}\vec{c}$};
				\draw[color=magenta,<-] (0,0,4) -- (3,1,8) node[above=5pt,right]{$-\vec{c}$};
				
				\end{tikzpicture}
			\end{flushleft}
			\column{.45\textwidth}
			\uncover<5->{$r \cdot \vec{c} = \left( \begin{array}{c} r \cdot x \\ r \cdot y \\ r \cdot z \end{array} \right)$}\\
			\ \\\uncover<6->{$ -\vec{c} = -1 \cdot \vec{c} = \left( \begin{array}{c} -x \\ -y \\ -z \end{array} \right)$}\\
			\ \\\uncover<7->{$ 2\vec{c} = \left( \begin{array}{c} 2 \cdot 4 \\ 2 \cdot 3 \\ 2 \cdot 1 \end{array} \right)$}
			\uncover<8->{$ = \left( \begin{array}{c} 8 \\ 6 \\ 2 \end{array} \right)$}\\
			\ \\\uncover<9->{$ -3\vec{c} = \left( \begin{array}{c} -3 \cdot 4 \\ -3 \cdot 3 \\ -3 \cdot 1 \end{array} \right)$}
			\uncover<10->{$ = \left( \begin{array}{c} -12 \\ -9 \\ -3 \end{array} \right)$}\\
			\ \\\uncover<11->{$ \frac{1}{2}\vec{c} = \left( \begin{array}{c} \frac{1}{2} \cdot 4 \\ \frac{1}{2} \cdot 3 \\ \frac{1}{2} \cdot 1 \end{array} \right)$}
			\uncover<12->{$ = \left( \begin{array}{c} 2 \\ 1,5 \\ 0,5 \end{array} \right)$}\\
		\end{columns}
	\end{frame}
	\begin{frame}{Aufgaben}
		\footnotesize
		$\vec{q} = \left( \begin{array}{c} 4 \\ 3 \\ 7 \end{array} \right)$
		$\vec{r} = \left( \begin{array}{c} 27 \\ 10 \\ 150 \end{array} \right)$
		$\vec{s} = \left( \begin{array}{c} 30 \\ 90 \\ 0,5 \end{array} \right)$
		$\vec{t} = \left( \begin{array}{c} 256 \\ 4 \\ 2 \end{array} \right)$
		Berechne.
		\begin{columns}[T]
			\column{0.35\textwidth}
			\begin{itemize}
				\item $\vec{q} + \vec{r} = $
				\uncover<2->{$\left( \begin{array}{c} 31 \\ 13 \\ 157 \end{array} \right)$}
				\item $\vec{q} - \vec{s} = $
				\uncover<2->{$\left( \begin{array}{c} -26 \\ -87 \\ 6,5 \end{array} \right)$}
				\item $\vec{t} - \vec{q} = $
				\uncover<2->{$\left( \begin{array}{c} 252 \\ 1 \\ -5 \end{array} \right)$}
				\item $-\vec{r} + \vec{s} = $
				\uncover<2->{$\vec{s} - \vec{r} = \left( \begin{array}{c} -226 \\ 86 \\ -1,5 \end{array} \right)$}
				\item $\vec{t} - \vec{t} = $				
				\uncover<2->{$\vec{o}$}
			\end{itemize}
			\column{0.65\textwidth}
			\begin{itemize}
				\item $25\vec{s} = $
				\uncover<2->{$\left( \begin{array}{c} 750 \\ 2250 \\ 12,5 \end{array} \right)$}
				\item $2\vec{q} = $
				\uncover<2->{$\left( \begin{array}{c} 8 \\ 6 \\ 14 \end{array} \right)$}
				\item $4\vec{q} - 3\vec{t}  = $
				\uncover<2->{$\left( \begin{array}{c} 16 \\ 12 \\ 28 \end{array} \right) - \left( \begin{array}{c} 768 \\ 12 \\ 6 \end{array} \right) = \left( \begin{array}{c} -752 \\ 0 \\ 22 \end{array} \right)$}
				\item $100\vec{q} - \vec{r} + \vec{s} = $
				\uncover<2->{$\left( \begin{array}{c} 400 \\ 300 \\ 700 \end{array} \right) + \left( \begin{array}{c} -226 \\ 86 \\ -1,5 \end{array} \right) = \left( \begin{array}{c} 174 \\ 386 \\ 698,5 \end{array} \right)$}
				\item $0\vec{t} =$				
				\uncover<2->{$\vec{o}$}
			\end{itemize}
			
		\end{columns}
	\end{frame}
	\lastslide
	\maketitle
	\section{Kollinearität}
	\begin{frame}{Kollinearität}
		\begin{columns}[T]
			\column{.5\textwidth}
			\begin{tikzpicture}[z=-.5cm]
				\draw<2->[color=red,->] (0,3,2) -- (6,4,5) node[left=10pt,below] {$\vec{v}_1$};
				\draw<2->[color=green,->] (9,4.5,6.5) -- (6,4,5) node[right=10pt,below] {$\vec{v}_2$};
			\end{tikzpicture}\\
			
			\uncover<3->{$\vec{v}_1 = \left( \begin{array}{c} 3 \\ 6 \\ 1 \end{array} \right)$}
			\uncover<3->{$\vec{v}_2 = \left( \begin{array}{c} -1,5 \\ -3 \\ -0,5 \end{array} \right)$}\\
			
			\begin{tikzpicture}[z=-.5cm]
			\draw<10->[color=red,->] (0,3,0) -- (9,6,3) node[left=10pt,below] {$\vec{v}_3$};
			\draw<10->[color=green,->] (6,5,3) -- (3,4,2) node[right=10pt,below] {$\vec{v}_4$};
			\end{tikzpicture}\\
			
			\begin{tikzpicture}[z=-.5cm]
			\draw<11->[color=blue,->] (6,5,3) -- (2,4,2) node[right=10pt,below] {$\vec{v}_5$};
			\draw<11->[color=black,->] (1,2,0) -- (4,4.5,2.5) node[right=5pt,below] {$\vec{v}_6$};
			\end{tikzpicture}\\
			
			\column{.5\textwidth}
			\uncover<4->{$\vec{v}_1, \vec{v}_2$ \textbf{kollinear}, wenn $\vec{v}_1 + r\cdot\vec{v}_2 = \vec{o}$}\\
			\ \\\uncover<5->{$\left( \begin{array}{c} x_1 \\ y_1 \\ z_1 \end{array} \right) + \left( \begin{array}{c} r \cdot x_2 \\ r \cdot y_2 \\ r \cdot z_2 \end{array} \right) = \vec{o}$}\\
			\ \\\uncover<6->{$\left( \begin{array}{c} x_1 + r \cdot x_2 \\ y_1 + r \cdot y_2 \\ z_1 + r \cdot z_2 \end{array} \right) = \left( \begin{array}{c} 0 \\ 0 \\ 0 \end{array} \right)$}\\
			\ \\\uncover<7->{$\left( \begin{array}{c} 3 + r \cdot (-1,5) \\ 6 + r \cdot (-3) \\ 1 + r \cdot (-0,5) \end{array} \right) = \left( \begin{array}{c} 0 \\ 0 \\ 0 \end{array} \right)$}\\
			\ \\\uncover<8->{$\begin{array}{l} 3 + r \cdot (-1,5) = 0 \\ 6 + r \cdot (-3) = 0 \\ 1 + r \cdot (-0,5) = 0 \end{array}$}
			\uncover<9->{$ \left| \begin{array}{l} r = -2 \\ r = -2 \\ r = -2 \end{array}\right. $}
		\end{columns}
	\end{frame}
	\begin{frame}{Aufgaben}
		Untersuchen Sie folgende Vektorpaare auf Kollinearität.
		\begin{columns}
			\column{.5\textwidth}
			\begin{itemize}
				\item $\vec{a} = \left( \begin{array}{c} 3 \\ 6 \\ 1 \end{array} \right) \vec{b} = \left( \begin{array}{c} -1,5 \\ -3 \\ -0,5 \end{array} \right)$
				\color{green}
				\uncover<2->\checkmark
				\color{black}
				\item $\vec{c} = \left( \begin{array}{c} 5 \\ 0 \\ 2 \end{array} \right) \vec{d} = \left( \begin{array}{c} 7 \\ 4 \\ 2 \end{array} \right)$
				\item $\vec{e} = \left( \begin{array}{c} 8 \\ 4 \\ 2 \end{array} \right) \vec{f} = \left( \begin{array}{c} 16 \\ 8 \\ 4 \end{array} \right)$
				\color{green}
				\uncover<2->\checkmark
				\color{black}
				\item $\vec{g} = \left( \begin{array}{c} 9 \\ 27 \\ 6 \end{array} \right) \vec{h} = \left( \begin{array}{c} -1 \\ -3 \\ -\frac{1}{3} \end{array} \right)$
				\color{green}
				\uncover<2->\checkmark
				\color{black}
			\end{itemize}
			\column{.6\textwidth}
			\begin{itemize}
				\item $\vec{j} = \left( \begin{array}{c} 4 \\ 28 \\ 1 \end{array} \right) \vec{k} = \left( \begin{array}{c} -1 \\ 7 \\ -0,25 \end{array} \right)$
				\item $\vec{l} = \left( \begin{array}{c} 6 \\ 6 \\ 6 \end{array} \right) \vec{m} = \left( \begin{array}{c} 6 \\ 6 \\ 6 \end{array} \right)$
				\color{green}
				\uncover<2->\checkmark
				\color{black}
				\item $\vec{o} = \left( \begin{array}{c} 3 \\ 5 \\ \sqrt{2} \end{array} \right) \vec{p} = \left( \begin{array}{c} 9 \\ 25 \\ 2 \end{array} \right)$
				\item $\vec{q} = \left( \begin{array}{c} 9k \\ 81 \\ 18 \end{array} \right) \vec{r} = \left( \begin{array}{c} -k \\ -9 \\  -2 \end{array} \right)$
				\color{green}
				\uncover<2->\checkmark
				\color{black}
			\end{itemize}
		\end{columns}
	\end{frame}
	\section{Linearkombination, Lineare (Un)Abhängigkeit}
	\begin{frame}{Linearkombination}
		\uncover<2->{$\vec{d}$ heißt \textbf{Linearkombination} von $\vec{a}, \vec{b}, \vec{c}$, wenn $\exists r,s,t \in\mathbb{R}$, sodass gilt:}\\
		\uncover<2->{$\vec{d} = r \cdot \vec{a} + s \cdot \vec{b} + t \cdot \vec{c}$}\\
		\ \\\uncover<3->{$\vec{a} = \left( \begin{array}{c} 4 \\ 1 \\ 6 \end{array} \right)$}
		\uncover<4->{$\vec{b} = \left( \begin{array}{c} 7 \\ 3 \\ 6 \end{array} \right)$}
		\uncover<5->{$\vec{c} = \left( \begin{array}{c} 16 \\ 2 \\ 5 \end{array} \right)$}\\
		\ \\\uncover<6->{$\vec{d}_1 = \vec{a} + \vec{b} + \vec{c}$}
		\uncover<7->{$ = \left( \begin{array}{c} 4 \\ 1 \\ 6 \end{array} \right) + \left( \begin{array}{c} 7 \\ 3 \\ 6 \end{array} \right) +\left( \begin{array}{c} 16 \\ 2 \\ 5 \end{array} \right)$}
		\uncover<8->{$ = \left( \begin{array}{l} 4 + 7 + 16 \\ 1 + 3 + 2 \\ 6 + 6 + 5 \end{array} \right)$}
		\uncover<9->{$ = \left( \begin{array}{c} 27 \\ 6 \\ 17 \end{array} \right)$}\\
		\ \\\uncover<10->{$\vec{d}_2 = 3 \cdot \vec{a} + 2 \cdot \vec{b} - 4 \cdot \vec{c}$}
		\uncover<11->{$ = 3\left( \begin{array}{c} 4 \\ 1 \\ 6 \end{array} \right) + 2\left( \begin{array}{c} 7 \\ 3 \\ 6 \end{array} \right) - 4\left( \begin{array}{c} 16 \\ 2 \\ 5 \end{array} \right)$}\\
		\ \\\uncover<12->{$\vec{d}_2 = \left( \begin{array}{l} 3 \cdot 4 + 2 \cdot 7 - 4 \cdot 16 \\ 3 \cdot 1 + 2 \cdot 3 - 3 \cdot 2 \\ 3 \cdot 6 + 2 \cdot 6 - 4 \cdot 5 \end{array} \right)$}
		\uncover<13->{$ = \left( \begin{array}{c} 12 + 14 - 64 \\ 3 + 6 - 6 \\  18 + 12 - 20 \end{array} \right)$}
		\uncover<14->{$ = \left( \begin{array}{c} -38 \\ 3 \\  10 \end{array} \right)$}
	\end{frame}
	\begin{frame}{Linearkombination}
		\footnotesize
		$\vec{a} = \left( \begin{array}{c} 4 \\ 1 \\ 6 \end{array} \right)$
		$\vec{b} = \left( \begin{array}{c} 7 \\ 3 \\ 6 \end{array} \right)$
		$\vec{c} = \left( \begin{array}{c} 16 \\ 2 \\ 5 \end{array} \right)$\ \ \ 
		$\vec{d}_3 = \left( \begin{array}{c} 5 \\ 2 \\ 3 \end{array} \right)$\\
		Ist $\vec{d}_3$ Linearkombinationen von $\vec{a}, \vec{b}, \vec{c}$?
		\color{green}
		\uncover<11->\checkmark
		\color{black}\\
		\ \\\uncover<2->{$\vec{d}_3 = r\cdot\vec{a} + s\cdot\vec{b} + t\cdot\vec{c}$}\\
		\ \\\uncover<3->{$\left( \begin{array}{c} 5 \\ 2 \\ 3 \end{array} \right) = r \left( \begin{array}{c} 4 \\ 1 \\ 6 \end{array} \right) + s \left( \begin{array}{c} 7 \\ 3 \\ 6 \end{array} \right) + t \left( \begin{array}{c} 16 \\ 2 \\ 5 \end{array} \right)$}
		\uncover<4->{$\left| \begin{array}{l} 5 = 4r + 7s + 16t \\ 2 = r + 3s + 2t \\ 3 = 6r + 6s + 5t \end{array} \right.$}\\
		\ \\\ \\\uncover<5->{$\left( \begin{array}{ccc|c} 4 & 7 & 16 & 5 \\ 1 & 3 & 2 & 2 \\ 6 & 6 & 5 & 3 \end{array} \right)$}
		\uncover<6->{$\begin{array}{l} -4II \\ \\ -6II \end{array}$}
		\uncover<7->{$\rightarrow\left( \begin{array}{ccc|c} 1 & 3 & 2 & 2 \\ 0 & -5 & 8 & -3 \\ 0 & -12 & -7 & -9 \end{array} \right)$}
		\uncover<8->{$\begin{array}{l} \\ \\ -\frac{12}{5}II \end{array}$}
		\uncover<9->{$\rightarrow\left( \begin{array}{ccc|c} 1 & 3 & 2 & 2 \\ 0 & -5 & 8 & -3 \\ 0 & 0 & -\frac{131}{5} & -\frac{9}{5} \end{array} \right)$}\\
		\uncover<10->{$\rightarrow \dots \rightarrow \left( \begin{array}{ccc|c} 1 & 0 & 0 & -\frac{35}{131} \\ 0 & 1 & 0 & \frac{93}{131} \\ 0 & 0 & 1 & \frac{9}{131} \end{array} \right)$}
	\end{frame}
	\begin{frame}{Lineare (Un)Abhängigkeit}
		\uncover<2->{$\vec{a}, \vec{b}, \vec{c}$ heißen \textbf{Linear unabhängig}, wenn $\exists r,s,t \in\mathbb{R}$, sodass}\\
		\uncover<2->{$r \cdot \vec{a} + s \cdot \vec{b} + t \cdot \vec{c} = \vec{o}$ nur für $r = s = t = 0$ lösbar ist.}\\
		\ \\\uncover<3->{\textbf{ODER}:}\\
		\ \\\uncover<3->{Keiner der Vektoren lässt sich als Linearkombination der übrigen darstellen.}\\
		\ \\\uncover<4->{zu Prüfen:}
		\begin{itemize}
			\item<4->ist $\vec{a}$ Linearkombination von $\vec{b}, \vec{c}$?
			\item<4->ist $\vec{b}$ Linearkombination von $\vec{c}, \vec{a}$?
			\item<4->ist $\vec{c}$ Linearkombination von $\vec{a}, \vec{b}$?
		\end{itemize}
		\ \\\ \\\uncover<5->{Anderenfalls heißen die Vektoren $\vec{a}, \vec{b}, \vec{c}$ \textbf{linear abhängig}.}
	\end{frame}
	\begin{frame}{Aufgaben}
		$\vec{q} = \left( \begin{array}{c} 4 \\ 3 \\ 7 \end{array} \right)$
		$\vec{r} = \left( \begin{array}{c} 550 \\ 620 \\ 150 \end{array} \right)$
		$\vec{s} = \left( \begin{array}{c} 5 \\ 1 \\ 0 \end{array} \right)$
		$\vec{t} = \left( \begin{array}{c} 3 \\ 4 \\ 1 \end{array} \right)$\\
		\ \\
		\begin{columns}[T]
			\column{0.5\textwidth}
			Sind $\vec{q}, \vec{r}, \vec{s}$ linear unabhängig?
			\color{green}
			\uncover<4->\checkmark
			\color{black}
			\begin{itemize}
				\item<2->ist $\vec{q}$ Linearkombination von $\vec{r}, \vec{s}$?
				\color{red}
				\uncover<3->x
				\color{black}
				\item<2->ist $\vec{r}$ Linearkombination von $\vec{s}, \vec{q}$?
				\color{red}
				\uncover<3->x
				\color{black}
				\item<2->ist $\vec{s}$ Linearkombination von $\vec{q}, \vec{r}$?
				\color{red}
				\uncover<3->x
				\color{black}
			\end{itemize}
			\column{0.5\textwidth}
			Sind $\vec{r}, \vec{s}, \vec{t}$ linear unabhängig?
			\color{red}
			\uncover<7->x
			\color{black}
			\begin{itemize}
				\item<5->ist $\vec{r}$ Linearkombination von $\vec{s}, \vec{t}$?
				\color{green}
				\uncover<6->\checkmark
				\color{black}
				\item<5->ist $\vec{s}$ Linearkombination von $\vec{t}, \vec{r}$?
				\item<5->ist $\vec{t}$ Linearkombination von $\vec{r}, \vec{s}$?
			\end{itemize}
		\end{columns}
	\end{frame}
	\lastslide
	\maketitle
	\section{Skalarprodukt}
	\begin{frame}{Skalarprodukt - Berechnung}
		\begin{columns}[t]
			\column{.5\textwidth}
			\uncover<3->{$\vec{a} \cdot \vec{b}$ bzw. $\vec{a} \circ \vec{b}$ ist eine Reelle Zahl $c$, für die gilt:\\}
			\uncover<3->{$c = | \vec{a} | \cdot | \vec{b} | \cdot \cos \sphericalangle ( \vec{a}, \vec{b} ) $\\}
			\ \\\uncover<2->{$c = x_a \cdot x_b + y_a \cdot y_b + z_a \cdot z_b$\\}
			\ \\\uncover<4->{\textbf{Eigenschaften:}\\}
			\begin{itemize}
				\item<5->$\vec{a} \cdot \vec{b} = \vec{b} \cdot \vec{a}$
				\item<6->$\vec{a} \cdot \left( \vec{b} + \vec{c} \right) = \left( \vec{a} + \vec{b} \right)\cdot \left( \vec{a} + \vec{c} \right)$
				\item<7->$r\left( \vec{a} \cdot \vec{b} \right) = \left( r\vec{a} \right) \cdot \vec{b} = \vec{a} \cdot \left( r\vec{b} \right)$
				\item<8->$\vec{a} \bot \vec{b} \Rightarrow \vec{a} \cdot \vec{b} = 0$
			\end{itemize}
			\column{.5\textwidth}
			\uncover<9->{\textbf{Beispiel:}\\}
			\uncover<10->{$\vec{a} = \left( \begin{array}{c} 2 \\ 4 \\ 5 \end{array} \right)$}
			\uncover<10->{$\vec{b} = \left( \begin{array}{c} 6 \\ 3 \\ 0 \end{array} \right)$\\}
			\ \\\uncover<11->{$c = 2 \cdot 6 + 4 \cdot 3 + 5 \cdot 0$}
			\uncover<12->{$ = 12 + 12 + 0$}
			\uncover<13->{$ = 24$\\}
			\ \\\uncover<14->{Stehen $\vec{a}$ und $\vec{b}$ senkrecht zueinander?\\}
			\uncover<15->{Nein, denn $\vec{a} \cdot \vec{b} \neq 0$}
		\end{columns}
	\end{frame}
	\subsection{Winkel zwischen Vektoren}
	\begin{frame}{Winkel zwischen Vektoren}
		\begin{columns}[t]
			\column{.45\textwidth}
			\uncover{$\vec{a} \cdot \vec{b}$ bzw. $\vec{a} \circ \vec{b}$ ist eine Reelle Zahl $c$, für die gilt:\\}
			\uncover{$c = | \vec{a} | \cdot | \vec{b} | \cdot \cos \sphericalangle ( \vec{a}, \vec{b} ) $\\}
			\ \\\uncover<2->{$\cos \sphericalangle ( \vec{a}, \vec{b} ) = \dfrac{c}{| \vec{a} | \cdot | \vec{b} |}$\\}
			\ \\\ \\\uncover<3->{$\cos \sphericalangle ( \vec{a}, \vec{b} ) = \dfrac{\vec{a} \cdot \vec{b}}{| \vec{a} | \cdot | \vec{b} |}$\\}
			\ \\\ \\\uncover<4->{$\sphericalangle ( \vec{a}, \vec{b} ) = \cos^{-1} \left( \dfrac{\vec{a} \cdot \vec{b}}{| \vec{a} | \cdot | \vec{b} |}  \right)$\\}
			\column{.55\textwidth}
			\uncover<5->{\textbf{Beispiel:}\\}
			\uncover<6->{$\vec{a} = \left( \begin{array}{c} 2 \\ 4 \\ 5 \end{array} \right)$}
			\uncover<6->{$\vec{b} = \left( \begin{array}{c} 6 \\ 3 \\ 0 \end{array} \right)$\\}
			\ \\\uncover<6->{$c = 2 \cdot 6 + 4 \cdot 3 + 5 \cdot 0 = 12 + 12 + 0 = 24$\\}
			\ \\\uncover<7->{$\sphericalangle ( \vec{a}, \vec{b} )$}
			\uncover<8->{$ = \cos^{-1} \left( \dfrac{24}{\sqrt{2^2 + 4^2 + 5^2} \cdot \sqrt{6^2 + 3^2}}  \right)$}
			\ \\\ \\\uncover<9->{$\sphericalangle ( \vec{a}, \vec{b} ) = \cos^{-1} \left( \dfrac{24}{\sqrt{45} \cdot \sqrt{45}}  \right)$}
			\uncover<10->{$ = \cos^{-1} \left( \dfrac{24}{45}  \right)$}
			\ \\\ \\\uncover<11->{$\sphericalangle ( \vec{a}, \vec{b} ) = 57,769\degree$}
		\end{columns}
	\end{frame}
	\begin{frame}{Aufgaben}
		$\vec{p} = \left( \begin{array}{c} 2 \\ 7 \\ 5 \end{array} \right)$
		$\vec{q} = \left( \begin{array}{c} 16 \\ 8 \\ -5 \end{array} \right)$
		$\vec{r} = \left( \begin{array}{c} 4 \\ 3 \\ 10 \end{array} \right)$
		$\vec{s} = \left( \begin{array}{c} \frac{1}{4} \\ \frac{1}{2} \\ 4 \end{array} \right)$\\\ \\
		\begin{columns}[t]
			\column{.5\textwidth}
			Berechnen Sie.
			\begin{itemize}
				\item $ \vec{p} \cdot \vec{q} = $
				\uncover<2->{$63$}
				\item $ \vec{q} \cdot \vec{r} = $
				\uncover<2->{$38$}
				\item $ \vec{r} \cdot \vec{s} = $
				\uncover<2->{$42,5$}
				\item $ 5 \left(\vec{q} \cdot \vec{r} \right) = $
				\uncover<2->{$190$}
				\item $ 5\vec{r} \cdot \vec{q} = $
				\uncover<2->{$190$}
				\item $ 20 \vec{q} \cdot 20 \vec{r} = $
				\uncover<2->{$400 \left( \vec{q} \cdot \vec{r} \right) = 15200$}
				\item $ 100 \vec{q} \cdot 20 \vec{r} = $
				\uncover<2->{$20 \left( 5\vec{q} \cdot \vec{r} \right) = 190$}
			\end{itemize}
			\column{.5\textwidth}
			Stehen die folgenden Vektorpaare senkrecht zueinander?
			\begin{itemize}
				\item $\vec{p}, \vec{r}$
				\color{red}
				\uncover<2->x
				\color{black}
				\item $\vec{q}, \vec{s}$
				\color{green}
				\uncover<2->\checkmark
				\color{black}
				\item $\vec{p}, \vec{s}$
				\color{red}
				\uncover<2->x
				\color{black}
			\end{itemize}
			Berechnen Sie.
			\begin{itemize}
				\item $\sphericalangle ( \vec{p}, \vec{q} ) \approx$
				\uncover<2->{$67,415\degree$}
				\item $\sphericalangle ( \vec{p}, \vec{r} ) \approx$
				\uncover<2->{$42,328\degree$}
				\item $\sphericalangle ( \vec{r}, \vec{s} ) \approx$
				\uncover<2->{$19,638\degree$}
			\end{itemize}
		\end{columns}
	\end{frame}
	\lastslide
	\maketitle
	\section{Vektorprodukt / Kreuzprodukt}
	\begin{frame}{Berechnung}
		
	\end{frame}
	\subsection{Flächeninhalt von Parallelogrammen / Dreiecken}
	\begin{frame}{Flächeninhalt von Parallelogrammen}
		
	\end{frame}
	\subsection{Flächeninhalt von Dreiecken}
	\begin{frame}{Flächeninhalt von Dreiecken}
		
	\end{frame}
	\lastslide
	\maketitle
	\section{Spatprodukt}
	\subsection{Spat? WTF?}
	\begin{frame}{Spat? WTF?}
		
	\end{frame}
	\subsection{Volumen eines Spates}
	\begin{frame}{Volumen eines Spates}
		
	\end{frame}
	\lastslide
\end{document}