\documentclass[11pt, aspectratio=169]{beamer}
\usetheme{default}

\usepackage[utf8]{inputenc}
\usepackage[german]{babel}
\usepackage{graphicx}
\usepackage{tikz}
\usepackage[absolute,overlay]{textpos}

\setlength{\TPHorizModule}{1cm}
\setlength{\TPVertModule}{1cm}

%\author{JeanHilftDir}
\title{Mathe Nachhilfe}
\subtitle{Analytische Geometrie - Einführung}
%\logo{}
%\institute{}
\date{}
%\subject{}
%\setbeamercovered{transparent}
\setbeamertemplate{navigation symbols}{}

\definecolor{lightblue}{RGB}{55,96,146}
\newcommand{\coord}{
		% draw help lines
		\foreach \x in {.5,1,1.5,2,2.5,3,3.5,4,4.5,5}
			\draw[style=help lines] (\x,0,0) -- (\x,0,5);
		\foreach \x in {.5,1,1.5,2,2.5,3,3.5,4,4.5,5}
			\draw[style=help lines] (\x,0,0) -- (\x,5,0);
		\foreach \y in {.5,1,1.5,2,2.5,3,3.5,4,4.5,5}
			\draw[style=help lines] (0,\y,0) -- (0,\y,5);
		\foreach \y in {.5,1,1.5,2,2.5,3,3.5,4,4.5,5}
			\draw[style=help lines] (0,\y,0) -- (5,\y,0);
		\foreach \z in {.5,1,1.5,2,2.5,3,3.5,4,4.5,5}
			\draw[style=help lines] (0,0,\z) -- (0,5,\z);
		\foreach \z in {.5,1,1.5,2,2.5,3,3.5,4,4.5,5}
			\draw[style=help lines] (0,0,\z) -- (5,0,\z);
		
		% draw Axes
		\draw[thick,->] (0,0,0) -- (5,0,0) node[below]{$y$};
		\draw[thick,->] (0,0,0) -- (0,5,0) node[left]{$z$};
		\draw[thick,->] (0,0,0) -- (0,0,5) node[left]{$x$};
		
		% draw labels
		\draw (0,0,0) -- (0,-.1,0) node[below]{$\scriptstyle0$};
		\foreach \x in {1,2,3,4}
			\draw (\x,-.1,0) -- (\x,.1,0) node[below=4pt] {$\scriptstyle\x$};
		\foreach \y in {1,2,3,4}
			\draw (-.1,\y,0) -- (.1,\y,0) node[left=4pt] {$\scriptstyle\y$};
		\foreach \z in {1,2,3,4}
			\draw (-.1,.1,\z) -- (.1,-.1,\z) node[right,below] {$\scriptstyle\z$};
	}
\newcommand{\lastslide}{
		\begin{frame}
			\only<2->{
				\begin{textblock}{5}(0.5,3.5)
					\includegraphics[width=5cm]{graphix/subpointer.png}
				\end{textblock}
			}
			\only<3->{
				\begin{textblock}{7}(8.5,0)
					\begin{flushright}
						\color{lightblue}\Large
						facebook.com/JeanHilftDir\\
						Skype: JeanHilftDir
					\end{flushright}
				\end{textblock}
				\begin{textblock}{9}(6.5,4)
					\begin{flushright}
						\color{lightblue}\Large
						Folien: github.com/JeanHilftDir/Mathe\\
						
					\end{flushright}
				\end{textblock}
			}
		\end{frame}
	}
\AtBeginSection[]{
		\begin{frame}{Analytische Geometrie - Einführung}
			\tableofcontents[currentsection]
		\end{frame}
	}

\begin{document}
	\maketitle
	\begin{frame}{Analytische Geometrie - Einführung}
		\tableofcontents
	\end{frame}
	\section{Ein neues Koordinatensystem}
	\subsection{Punkte ablesen}
	\begin{frame}{Ein neues Koordinatensystem - Punkte ablesen}
		\begin{columns}[T]
			\column{.7\textwidth}
			\begin{tikzpicture}[z=-.5cm]
			\coord
			
			\coordinate<2->[label=right:$P$] (P) at (1.5,1.5,0);
			\fill<2-> (P) circle (2pt);
			
			\draw<3->[color=blue, line width=2pt] (0,0,0) --(1.5,0,0);
			\draw<4->[color=blue, line width=2pt] (1.5,0,0) --(1.5,1.5,0);
			
			\draw<6->[color=green, line width=2pt] (0,0,0) --(0,0,1);
			\draw<7->[color=green, line width=2pt] (0,0,1) --(2,0,1);
			\draw<8->[color=green, line width=2pt] (2,0,1) --(2,2,1);
			
			\draw<10->[color=red, line width=2pt] (0,0,0) --(0,0,3);
			\draw<11->[color=red, line width=2pt] (0,0,3) --(3,0,3);
			\draw<12->[color=red, line width=2pt] (3,0,3) --(3,3,3);
			
			
			
			\end{tikzpicture}
			
			\column{.3\textwidth}
			\uncover<5->{$P(0,1.5,1.5)$}\\
			\uncover<9->{$P(1,2,2)$}\\
			\uncover<13->{$P(3,3,3)$}
		\end{columns}
	\end{frame}
	\subsection{Punkte eintragen}
	\begin{frame}{Ein neues Koordinatensystem - Punkte eintragen}
		\begin{columns}[T]
			\column{.7\textwidth}
			\begin{tikzpicture}[z=-.5cm]
				\coord
				
				
				\draw<12->[color=red, line width=2pt] (0,0,3) --(5,0,3);
				\draw<15->[color=red, line width=2pt] (0,0,0) --(5,0,0);
				\draw<15->[color=red, line width=2pt] (5,0,0) --(5,0,3);
				
				\draw<3->[color=blue, line width=2pt] (0,0,0) --(0,0,5);
				\draw<4->[color=blue, line width=2pt] (0,0,5) --(3,0,5);
				\draw<5->[color=blue, line width=2pt] (3,0,5) --(3,6,5);
				
				\draw<7->[color=green, line width=2pt] (0,0,0) --(0,0,4);
				\draw<8->[color=green, line width=2pt] (0,0,4) --(2,0,4);
				\draw<9->[color=green, line width=2pt] (2,0,4) --(2,1,4);
				
				\draw<11->[color=red, line width=2pt] (0,0,0) --(0,0,3);
				\draw<13->[color=red, line width=2pt] (5,0,3) --(5,1,3);
				
				\coordinate<6->[label=right:$P$] (P) at (3,6,5);
				\fill<6-> (P) circle (2pt);
				
				\coordinate<10->[label=right:$Q$] (P) at (2,1,4);
				\fill<10-> (P) circle (2pt);
				
				\coordinate<14->[label=right:$R$] (P) at (5,1,3);
				\fill<14-> (P) circle (2pt);
			
			\end{tikzpicture}
			
			\column{.3\textwidth}
			\uncover<2->{$P(5,3,6)$}\\
			\uncover<2->{$Q(4,2,1)$}\\
			\uncover<2->{$R(3,5,1)$}
		\end{columns}
	\end{frame}
	\lastslide
	\section{Vektoren}
	\subsection{Ortsvektoren}
	\begin{frame}{Vektoren - Ortsvektoren}
		\begin{columns}[T]
			\column{.7\textwidth}
			\begin{tikzpicture}[z=-.5cm]
			
			\coord
			
			\coordinate<2->[label=right:$A$] (A) at (5,1,3);
			\fill<2-> (A) circle (1pt);
			
			\draw<3->[->] (0,0,0) -- (5,1,3) node[right, below=1pt] {$\vec{v}_1 = \left( \begin{array}{c} 3 \\ 5 \\ 1 \end{array} \right)$};
			
			\coordinate<4->[label=above:$B$] (B) at (6,3,4);
			\fill<4-> (B) circle (1pt);
			
			\draw<5->[->] (0,0,0) -- (6,3,4) node[right] {$\vec{v}_2 = \left( \begin{array}{c} 4 \\ 6 \\ 3 \end{array} \right)$};
			
			\coordinate<6->[label=right:$C$] (C) at (-2,4,1);
			\fill<6-> (C) circle (1pt);
			
			\draw<7->[->] (0,0,0) -- (-2,4,1) node[right=30pt, above=-15pt] {$\vec{v}_3 = \left( \begin{array}{c} 1 \\ -2 \\ 4 \end{array} \right)$};
			
			\coordinate<8->[label=right:$P$] (P) at (2,-.5,3.5);
			\fill<8-> (P) circle (1pt);
			
			\coordinate<9->[label=above:$Q$] (Q) at (0,3.5,4.5);
			\fill<9-> (Q) circle (1pt);
			
			\draw<10->[->] (2,-.5,3.5) -- (0,3.5,4.5) node[below=70pt,right=50pt] {$\vec{v}_4 = \left( \begin{array}{c} 1 \\ -2 \\ 4 \end{array} \right)$};
			
			\end{tikzpicture}
			
			\column{.3\textwidth}
			\uncover<2->{$A=(3,5,1)$}\\
			\uncover<4->{$B=(4,6,3)$}\\
			\uncover<6->{$C=(1,-2,4)$}\\
			\uncover<8->{$P=(3.5,2,-0.5)$}\\
			\uncover<9->{$Q=(4.5,0,3.5)$}\\
			\ \\\uncover<11->{$\vec{v}_1 = \vec{OA}$}\\
			\uncover<12->{$\vec{v}_2 = \vec{OB}$}\\
			\uncover<13->{$\vec{v}_3 = \vec{OC}$}\\
			\ \\\uncover<14->{$\vec{v}_4 = \vec{PQ} = \vec{v}_3$}
		\end{columns}
	\end{frame}
	\subsection{Vektoren durch zwei Punkte}
	\begin{frame}{Vektoren durch zwei Punkte}
		\begin{columns}[T]
			\column{.55\textwidth}
			\begin{tikzpicture}[z=-.5cm]
			
			\coord
			
			\coordinate[label=right:$P$] (P) at (2,-.5,3.5);
			\fill (P) circle (1pt);
			
			\coordinate[label=above:$Q$] (Q) at (0,3.5,4.5);
			\fill (Q) circle (1pt);
			
			\draw[->] (2,-.5,3.5) -- (0,3.5,4.5) node[below=70pt,right=50pt] {$\vec{v}_4$};
			
			\end{tikzpicture}
			
		\column{.45\textwidth}
		\uncover{$P=(3.5,2,-0.5)$}\\
		\uncover{$Q=(4.5,0,3.5)$}\\
		\ \\\uncover<2->{$\vec{v}_4 = \vec{PQ} = \vec{OQ} - \vec{OP}$}\\
		\uncover<3->{$\vec{v}_4 = \left( \begin{array}{c} x_q - x_p \\ y_q - y_p \\ z_q - z_p \end{array} \right)$}\\
		\uncover<4->{$\vec{v}_4 = \left( \begin{array}{c} 4.5 - 3.5 \\ 0 - 2 \\ 3.5 - (-0.5) \end{array} \right)$}
		\uncover<5->{$ = \left( \begin{array}{c} 1 \\ -2 \\ 4 \end{array} \right)$}
	\end{columns}
	\end{frame}
	\lastslide
	\section{Länge eines Vektors}
	\subsection{2D}
	\begin{frame}{Länge eines Vektors - 2D}
		\begin{columns}[T]
			\column{.6\textwidth}
			\begin{tikzpicture}[z=-.5cm]
			
			\coord
			
			\coordinate<2->[label=below:$R$] (R) at (3,2);
			\fill<2-> (R) circle (1pt);
			
			\coordinate<3-> (S) at (4,5);
			\draw<3-> (S) node[right=5pt,below]{$S$};
			\fill<3-> (S) circle (1pt);
			
			\draw<4->[->] (3,2) -- (4,5) node[left=10pt,below] {$\vec{v}_5$};
			
			\draw<5->[thick,color=blue] (3,2) -- (4,2);
			\draw<6->[thick,color=blue] (4,2) -- (4,5);
			
			\end{tikzpicture}
			
			\column{.4\textwidth}
			\uncover<2->{$R=(3,2)$}\\
			\uncover<3->{$S=(4,5)$}\\
			\uncover<7->{$\vec{v}_5 = \left( \begin{array}{c} 4 - 3 \\ 5 - 2 \end{array} \right)$}
			\uncover<8->{$ = \left( \begin{array}{c} 1 \\ 3 \end{array} \right)$}\\
			\uncover<9->{$\left| \vec{v}_5 \right| = \sqrt{x^2+y^2}$}\\
			\uncover<10->{$\left| \vec{v}_5 \right| = \sqrt{1^2+3^2}$}
			\uncover<11->{$ = \sqrt{10}$}\\
			\uncover<12->{$\left| \vec{v}_5 \right| \approx 3.1623 $}
			
		\end{columns}
	\end{frame}
	\subsection{3D}
	\begin{frame}{Länge eines Vektors - 3D}
		\begin{columns}[T]
			\column{.55\textwidth}
			\begin{tikzpicture}[z=-.5cm]
			
			\coord
			
			\draw<7>[thick,color=red] (2,-.5,1.5) -- (-.5,-.5,1.5);
			\draw<7>[thick,color=red] (-.5,-.5,1.5) -- (0,3.5,2.5);
			
			\draw<8,19>[thick,color=red] (2,-.5,1.5) -- (0,-.5,1.5);
			\draw<8,19>[thick,color=red] (0,-.5,1.5) -- (0,3.5,2.5);
			
			\draw<11->[thick,color=green] (2,-.5,2.5) -- (0,-.5,2.5);
			\draw<11->[thick,color=green] (0,-.5,2.5) -- (0,3.5,2.5);
			
			\draw<10->[thick,color=blue] (2,-.5,1.5) -- (2,-.5,2.5);
			\draw<10->[thick,color=blue] (2,-.5,2.5) -- (0,3.5,2.5);
			
			\coordinate<2->[label=right:$U$] (U) at (2,-.5,1.5);
			\fill<2-> (U) circle (1pt);
			
			\coordinate<3->[label=above:$V$] (V) at (0,3.5,2.5);
			\fill<3-> (V) circle (1pt);
			
			\draw<4->[->] (2,-.5,1.5) -- (0,3.5,2.5) node[below,right] {$\vec{v}_6$};
			
			
			\end{tikzpicture}
			
			\column{.45\textwidth}
			\uncover<2->{$U=(1.5,2,-0.5)$}\\
			\uncover<3->{$V=(2.5,0,3.5)$}\\
			
			\uncover<5->{$\vec{v}_6 = \left( \begin{array}{c} 2.5 - 1.5 \\ 0 - 2 \\ 3.5 - (-0,5) \end{array} \right)$}
			\uncover<6->{$ = \left( \begin{array}{c} 1 \\ -2 \\ 4\end{array} \right)$}\\
			\uncover<12->{$\left| \vec{v}_6 \right| = \sqrt{x^2+\left| \left( \begin{array}{c} 0 \\ y \\ z\end{array} \right) \right|^2}$}\\
			\uncover<13->{$\left| \vec{v}_6 \right| = \sqrt{x^2+\sqrt{y^2+z^2}^2}$}\\
			\uncover<14->{$\left| \vec{v}_6 \right| = \sqrt{x^2+y^2+z^2}$}\\
			\ \\\uncover<15->{$\left| \vec{v}_6 \right| = \sqrt{1^2+(-2)^2+4^2}$}\\
			\uncover<16->{$\left| \vec{v}_6 \right| = \sqrt{1+4+16}$}
			\uncover<17->{$ = \sqrt{21}$}
			\uncover<18->{$\approx 4.5826$}
			
		\end{columns}
	\end{frame}
	\lastslide
\end{document}