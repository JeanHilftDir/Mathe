\documentclass[11pt, aspectratio=169]{beamer}
\usetheme{default}

\usepackage[utf8]{inputenc}
\usepackage[german]{babel}
\usepackage{graphicx}
\usepackage{tikz}
\usepackage[absolute,overlay]{textpos}

\setlength{\TPHorizModule}{1cm}
\setlength{\TPVertModule}{1cm}

%\author{JeanHilftDir}
\title{Mathe Nachhilfe}
\subtitle{Analytische Geometrie - Vektoren}
%\logo{}
%\institute{}
\date{}
%\subject{}
%\setbeamercovered{transparent}
\setbeamertemplate{navigation symbols}{}

\definecolor{lightblue}{RGB}{55,96,146}
\newcommand{\coord}{
		% draw help lines
		%\foreach \x in {.5,1,1.5,2,2.5,3,3.5,4,4.5,5}
		%	\draw[style=help lines] (\x,0,0) -- (\x,0,5);
		%\foreach \x in {.5,1,1.5,2,2.5,3,3.5,4,4.5,5}
		%	\draw[style=help lines] (\x,0,0) -- (\x,5,0);
		%\foreach \y in {.5,1,1.5,2,2.5,3,3.5,4,4.5,5}
		%	\draw[style=help lines] (0,\y,0) -- (0,\y,5);
		%\foreach \y in {.5,1,1.5,2,2.5,3,3.5,4,4.5,5}
		%	\draw[style=help lines] (0,\y,0) -- (5,\y,0);
		%\foreach \z in {.5,1,1.5,2,2.5,3,3.5,4,4.5,5}
		%	\draw[style=help lines] (0,0,\z) -- (0,5,\z);
		%\foreach \z in {.5,1,1.5,2,2.5,3,3.5,4,4.5,5}
		%	\draw[style=help lines] (0,0,\z) -- (5,0,\z);
		
		% draw Axes
		\draw[thick,->] (0,0,0) -- (5,0,0) node[below]{$y$};
		\draw[thick,->] (0,0,0) -- (0,5,0) node[left]{$z$};
		\draw[thick,->] (0,0,0) -- (0,0,5) node[left]{$x$};
		
		% draw labels
		\draw (0,0,0) -- (0,-.1,0) node[below]{$\scriptstyle0$};
		\foreach \x in {1,2,3,4}
			\draw (\x,-.1,0) -- (\x,.1,0) node[below=4pt] {$\scriptstyle\x$};
		\foreach \y in {1,2,3,4}
			\draw (-.1,\y,0) -- (.1,\y,0) node[left=4pt] {$\scriptstyle\y$};
		\foreach \z in {1,2,3,4}
			\draw (-.1,.1,\z) -- (.1,-.1,\z) node[right,below] {$\scriptstyle\z$};
	}
\newcommand{\lastslide}{
		\begin{frame}
			\only<2->{
				\begin{textblock}{5}(0.5,3.5)
					\includegraphics[width=5cm]{graphix/subpointer.png}
				\end{textblock}
			}
			\only<3->{
				\begin{textblock}{7}(8.5,0)
					\begin{flushright}
						\color{lightblue}\Large
						facebook.com/JeanHilftDir\\
						Skype: JeanHilftDir
					\end{flushright}
				\end{textblock}
				\begin{textblock}{9}(6.5,4)
					\begin{flushright}
						\color{lightblue}\Large
						Folien: github.com/JeanHilftDir/Mathe\\
						
					\end{flushright}
				\end{textblock}
			}
		\end{frame}
	}
\AtBeginSection[]{
		\begin{frame}{Analytische Geometrie - Vektoroperationen}
			\tableofcontents[currentsection]
		\end{frame}
	}

\begin{document}
	\maketitle
	\begin{frame}{Analytische Geometrie - Vektoroperationen}
		\tableofcontents
	\end{frame}
	\section{Begriffe}
	\subsection{Null-, Einheits-, Gegenvektor, Kartesisches Koordinatensystem}
	\begin{frame}{Nullvektor, Einheitsvektor, Entgegengesetzter Vektor}
		\textbf{Nullvektor $\vec{o}$:}
		
		\textbf{Einheitsvektor:}
		
		\textbf{Entgegengesetzter Vektor zu $\vec{a}$:}
		
		
	\end{frame}
	\subsection{Kollinearität, Linearkombination, Lineare (Un)Abhängigkeit}
	\lastslide
	\maketitle
	\section{Grundrechenarten}
	\subsection{Addition, Subtraktion}
	\begin{frame}{Addition, Subtraktion}
		\begin{columns}[T]
			\column{.7\textwidth}
			\begin{tikzpicture}[z=-.5cm]
			\coord
			
			\end{tikzpicture}
			
		\end{columns}
	\end{frame}
	\subsection{Multiplikation mit Skalar / Skalierung}
	\begin{frame}{}
		
	\end{frame}
	\lastslide
	\maketitle
	\section{Skalarprodukt}
	\begin{frame}{Berechnung}
		
	\end{frame}
	\subsection{Eigenschaften}
	\begin{frame}{Eigenschaften}
		
	\end{frame}
	\subsection{Winkel zwischen Vektoren}
	\begin{frame}{Winkel zwischen Vektoren}
		
	\end{frame}
	\lastslide
	\maketitle
	\section{Vektorprodukt / Kreuzprodukt}
	\begin{frame}{Berechnung}
		
	\end{frame}	
	\subsection{Eigenschaften}
	\begin{frame}{Eigenschaften}
		
	\end{frame}
	\subsection{Flächeninhalt von Parallelogrammen / Dreiecken}
	\begin{frame}{Flächeninhalt von Parallelogrammen}
		
	\end{frame}
	\subsection{Flächeninhalt von Dreiecken}
	\begin{frame}{Flächeninhalt von Dreiecken}
		
	\end{frame}
	\lastslide
	\maketitle
	\section{Spatprodukt}
	\subsection{Spat? WTF?}
	\begin{frame}{Spat? WTF?}
		
	\end{frame}
	\subsection{Volumen eines Spates}
	\begin{frame}{Volumen eines Spates}
		
	\end{frame}
	\lastslide
\end{document}